\chapter*{Introduzione}
\chaptermark{Introduzione}
\addcontentsline{toc}{chapter}{Introduzione}
\label{chap:introduction}

L’implementazione di strumenti atti a combattere gli attacchi di tipo \textbf{denial of service} (\textbf{DoS}), sono da molto
tempo materia di studio e di ricerca. Questi attacchi possono rappresentare una minaccia grave, soprattutto quando si
verificano su servizi cruciali e sensibili. Per tale motivo, negli anni sono state sviluppate molte strategie diverse
per impedire a singoli o ai gruppi di computer (noti anche come "zombie armies") di attaccare servizi e reti.
Attualmente una delle metodologie di protezione più comuni agli attacchi DoS è il \textbf{Rate-limiting}, che permette
di imporre una frequenza massima accettabile, per le richieste, al fine di evitare sovraccarichi e congestionamenti delle
risorse. In particolare, nei sistemi su larga scala, la limitazione della frequenza rappresenta uno strumento essenziale
per garantire la disponibilità e l'integrità delle risorse e dei servizi.

Un altro campo di interesse in cui si sono fatti molti progressi negli ultimi anni, grazie soprattutto all'avvento delle
tecnologie blockchain, è stato quello dell’anonimato in rete. Con il termine anonimato in rete ci si riferisce alla
condizione in cui, sulla base di una conoscenza parziale o totale delle interazioni di un utente in una rete, non è
possibile risalire all'identità dell'utente stesso; permettendo un'interazione con i servizi offerti dalla rete in
assoluta riservatezza. L'anonimato in rete presenta molteplici vantaggi, soprattutto in contesti in cui la privacy
costituisce un requisito fondamentale, come ad esempio nelle votazioni o nelle transazioni finanziarie. Inoltre, la
possibilità di mantenere l'anonimato può rivelarsi altrettanto utile in ambiti più diffusi, come nelle conversazioni online
o sui social network, garantendo la libertà di espressione e la tutela della propria sfera privata.

Tuttavia, l'anonimato in rete presenta anche alcune criticità, tra cui la principale è rappresentata dalla difficoltà di
controllo. La ragione di questa difficoltà è da individuare nel fatto che le metodologie di sicurezza a livello applicazione (pila ISO/OSI)
generalmente si basano sull' analisi del comportamento degli utenti o dei dispositivi, nel corso del tempo, al
fine di rilevare i pattern di attività che potrebbero suggerire la presenza di un attacco. In un contesto anonimo, in
cui le identità degli utenti non sono tracciabili, ciò diviene notevolmente più arduo. Esistono strumenti di
rate-limiting efficaci a livelli inferiori della pila ISO/OSI, come a livello di trasporto o di sessione. Tuttavia, la
loro attuazione comporta alcuni svantaggi, come il rischio di limitare o bloccare numerosi indirizzi IP tradotti sotto
una NAT, minare la privacy degli utenti disattivando protezioni utili come TLS o ancora essere elusi da
tecniche di mascheramento, come l'IP spoofing. Per tali motivazioni, per un servizio diffuso in ambiente anonimo, non è
possibile fare affidamento esclusivamente su questi strumenti.
\clearpage
La presente tesi affronta la discussione di un protocollo chiamato \textbf{RLN (Rate-Limiting Nullifier)}\footnote{\url{https://rate-limiting-nullifier.github.io/rln-docs/}} basato sulla tecnologia
\textbf{zk-SNARK}, che permette di attuare un rate-limiting in ambiente anonimo. Il protocollo è composto da tre parti generali,
le quali si differenziano, nei dettagli, a seconda del dominio applicativo.
Fasi del protocollo:
\begin{itemize}
    \item \textbf{Registrazione}: Durante questa fase, gli utenti che desiderano accedere al servizio devono
    registrarsi fornendo una prova del possesso di determinati requisiti. Questa prova di possesso è denominata
    "identity commitment" e viene conservata e utilizzata nelle fasi successive del protocollo. I dati personali che
    soddisfano i requisiti sono definiti "stake" e possono assumere forme diverse a seconda del contesto applicativo. Ad
    esempio, possono essere costituiti da un profilo su un social network, dall'indirizzo di un portafoglio di
    criptovalute o da un'identità digitale come lo SPID o la CIE.

    Gli stake non vengono conservati dal protocollo. Alla fine della fase di registrazione, il sistema è a conoscenza
    unicamente nel fatto che un nuovo utente anonimo che soddisfa le specifiche della registrazione è
    stato aggiunto al servizio. La presenza di uno stake non è strettamente necessaria per la registrazione.
    In effetti, è possibile registrarsi semplicemente creando un codice univoco e fornendolo come identity commitment.
    Tuttavia, l'utilizzo di uno stake risulta essere molto utile al fine di prevenire attacchi Sybil, ovvero la
    generazione di numerosi utenti malevoli da parte di un attaccante.
    \item \textbf{Interazione}: Dopo essersi registrati, gli utenti hanno la possibilità di interagire con il servizio
    attraverso l'invio di richieste. Ad ogni richiesta, gli utenti sono tenuti a fornire una prova di appartenenza al
    sistema. Tale prova è generata tramite la tecnologia zk-SNARK, che consente di dimostrare al servizio che l'utente è
    effettivamente un membro legittimo senza rivelare alcuna informazione sulla sua identità. Inoltre, il protocollo
    consente di implementare una regola di rate-limiting, che, se non rispettata, porta l'utente a rivelare la propria
    identità. Ciò consente di scoprire e gestire chi attua comportamenti di spam o DoS, e di procedere alla fase
    successiva.
    \item \textbf{Punizione}: La tipologia e il grado di punizione dipendono molto dal contesto applicativo. Ad esempio,
    alcune punizioni o misure di "slashing" possono comportare la rimozione del membro dal gruppo, con conseguente
    impossibilità di interagire con il sistema e perdita dell'anonimato. In presenza di una stake, è possibile
    prevedere sanzioni più articolate, come l'inserimento dell'identità virtuale in un registro di esclusione per altri
    servizi, o la rivelazione dell'indirizzo del portafoglio criptovalute dell'utente e il sequestro dei fondi.
\end{itemize}
Di seguito si cercherà di guardare a tutto tondo le tecnologie e le idee alla base di zk-SNARK e del protocollo RLN.
Inoltre, si mostrerà l'implementazione di un piccolo prototipo che utilizza questo protocollo per evitare il sovraccarico
di risorse in un servizio basato su API.
