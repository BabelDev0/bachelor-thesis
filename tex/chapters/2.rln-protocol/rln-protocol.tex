\chapter{Analisi del protocollo RLN}
\chaptermark{Analisi del protocollo RLN}
\label{chap:rln-protocol}

Nel presente capitolo, tratteremo la discussione del protocollo RLN (Rate Limiting Nullifier), che consente la
costruzione di regole di rate-limiting anti DoS e spam in ambiente anonimo utilizzando la tecnologia zk-SANRK. La prima proposta del protocollo è stata
sviluppata da Barry WhiteHat, un ricercatore attivo nel campo della blockchain e delle applicazioni Zero Knowledge, nel
seguente post: "Semaphore RLN, rate-limiting nullifier for spam prevention in anonymous p2p setting" \cite{semaphore-rln}. Attualmente, RLN
fa parte di (PSE) Privacy \& Scaling Explorations \cite{pse}, un team multidisciplinare sostenuto dalla Fondazione Ethereum, che
esplora nuove tecnologie Zero Knowledge e altre primitive crittografiche. Alcuni progetti di rilievo includono zk-kit \cite{zk-kit},
Interep \cite{interep} e Semaphore \cite{semaphore}. RLN è ancora un protocollo poco conosciuto, e non esistono ancora grosse implementazioni al di
fuori del contesto di ricerca. Il progetto in stato più avanzato è relativo a un lavoro portato avanti da Vac che sta
lavorando sull'implementazione del protocollo all'interno di Waku v2, la seconda versione di un protocollo di
comunicazione peer-to-peer privacy-preserving in ambiente decentralizzato.

Nelle prossime sezioni, verranno dettagliate le fasi del protocollo e le tecnologie utilizzate. Tuttavia, prima di
addentrarci, potrebbe essere utile fornire una descrizione generale e concisa del protocollo per comprendere meglio dove
e perché vengono impiegate le tecnologie che andremo a esaminare. Il protocollo RLN è suddiviso in tre fasi. Descriveremo brevemente le prime due, in quanto la terza diventa ovvia una volta compreso il funzionamento delle prime.

\textbf{Regitrazione}: Questa fase consente agli utenti di registrarsi al servizio che utilizza il protocollo RLN. In
particolare, agli utenti viene richiesto di generare una chiave privata che rappresenta il loro segreto. Le strategie
per la generazione della chiave possono variare a seconda del contesto applicativo e possono essere correlate alla
presenza o meno di una "stake". Una volta ottenuta la chiave privata, a questa viene applicata una funzione di hash per
generare un "identity commitment", ovvero un dato che può essere reso pubblico in quanto non consente di ricostruire la
chiave privata, ma che ne garantisce l'identificazione univoca. Durante questa fase, l'identity commitment generato
viene inviato al servizio, che lo inserisce in una particolare struttura dati chiamata albero di Merkle. In questo
albero, il sistema salva e mantiene tutti gli identity commitment degli utenti registrati. La registrazione rappresenta
una fase cruciale del protocollo, in quanto solo dopo di essa gli utenti saranno in grado di interagire con il sistema,
generando delle Zero Knowledge proof che attestino la loro appartenenza al servizio. Questa procedura è anche chiamata "proof
of membership". Al fine di limitare gli attacchi DoS o Sybil, la fase di registrazione sarà molto più efficace se
subordinata al rispetto di determinate specifiche, di difficile riproducibilità da parte degli utenti. Per questo, in
alcuni casi, è possibile legare la chiave privata a una stake.

\textbf{Interazione}: Questa fase rappresenta sia quella che consente l'interazione tra gli utenti registrati e il
sistema, sia quella che implementa la regola di rate limiting. La chiave privata dell'utente, prima che interagisca con
il sistema, viene suddivisa in $n$ parti in modo che ad ogni interazione l'utente riveli al sistema solo una porzione
della chiave. Il sistema non sarà in grado di ricostruire la chiave privata dell'utente dalle singole porzioni, a meno
che queste non raggiungano un numero $k$ stabilito dalla regola di rate limiting. In caso contrario, se il sistema
dispone di k parti della chiave, è in grado di ricavare l'identità dell'utente. Questo processo è attuabile grazie ad un
algoritmo basato sull'interpolazione di Lagrange, chiamato Shamir Secret Sharing (SSS).
\section{Strumenti}

\subsection{Alberi di Merkle}
Gli alberi di Merkle sono una particolare struttura ad albero che sfrutta le proprietà delle funzioni di Hash al fine di
ottimizzare le operazioni di ricerca e identificazione delle modifiche all'interno della collezione. La struttura degli
alberi di Merkle è caratterizzata dalle di foglie, che rappresentano i dati, a cui è stata applicata una funzione di
hashing, e dagli altri nodi dell'albero, che sono ottenuti applicando la funzione Hash ai loro figli, fino a raggiungere
la radice. Negli alberi di Merkle la radice è una funzione Hash che rappresenta univocamente la struttura. Per capire
meglio il concetto immaginiamo di avere un struttura ad albero binaria, e inseriamo all'interno della struttua gli
elementi 1,2,3,4,5 a cui applichimao una funzione Hash, a questo punto otterremo la seguente struttura:
\begin{figure}[H]
    \centering
    \includegraphics[width=15cm]{./chapters/2.rln-protocol/images/1.merkle_tree.png}
    \label{fig:merkle_tree}
    \captionsetup{justification=centering}
    \caption{Esempio struttura albero di Merkle binario}
\end{figure}
Possiamo notare che la radice dei Merkle tree possiede la proprietà di essere una sorta di impronta digitale della
struttura, in quanto qualsiasi modifica ai dati comporterebbe un cambiamento a cascata dei nodi fino alla radice stessa.
Questa proprietà costituisce un vantaggio significativo in termini di efficienza. Infatti, consente una verifica rapida
delle modifiche apportate alla struttura, rendendo gli alberi di Merkle estremamente utili nei campi decentralizzati e
nei file system condivisi, dove l'individuazione efficiente dei cambiamenti con poche informazioni è cruciale. Un'altra
proprietà altamente utile degli alberi di Merkle è la loro capacità di verificare l'appartenenza di un elemento alla
struttura in modo efficiente e senza la necesità di conoscere l'elemento in chairo. Nell'esempio precedente, se si
desidera verificare che l'elemento 4 appartenga alla struttura, è sufficiente utilizzare gli elementi H\_3, H\_12 e
H\_5000 e il valore H\_4, che non rivela alcuna informazion su l'elemento, per ricostruire tutti i nodi fino alla
radice. Una volta ottenuta la radice basterà confrontarla con la radice corretta per convicersi della presenza
dell'elemento o meno nella struttura.
\begin{figure}[H]
    \centering
    \includegraphics[width=15cm]{./chapters/2.rln-protocol/images/2.merkle_proof.png}
    \label{fig:merkle_proof}
    \captionsetup{justification=centering}
    \caption{Esempio Merkle proof}
\end{figure}
Questo processo viene chiamato "Merkle proof" o più in generale "proof of membership". Tale processo rappresenta
l'approccio che adotteremo per dimostrare la capacità di un utente già registrato e non rimosso di interagire con il
sistema nella fase di Interazione del protocollo RLN.

\subsection{Nuove funzioni di Hash}
Indubbiamente, una delle tecniche più utilizzate in crittografia sono le funzioni di Hash. Anche la tecnologia zk-SNARK
non può fare a meno di esse. Infatti, come abbiamo già visto in molte situazioni, l'utilizzo di metodi di hashing è
stato necessario per ridurre le dimensioni delle informazioni e per nasconderle.Tuttavia, quando si utilizzano le
tradizionali funzioni di Hash come la versione sha-256 nel campo delle zero knowledge proof, si verifica un problema.
Queste funzioni non sono state concepite per lavorare in domini di campi finiti e, pertanto, l'utilizzo di metodologie
come la ripetizione di operazioni bit-wise tende ad aumentare considerevolmente la dimensione dei vincoli del circuito
della prova che le implementa. Questo aumenta notevolmente il tempo e la dimensione richiesti per generare le prove.
Negli ultimi anni, per superare il problema descritto, sono state utilizzate nuove versioni di funzioni di Hash come
Pedersen, Mack e Poseidon. Queste funzioni sono state progettate specificamente per lavorare nei campi finiti e con
l'obiettivo di ridurre al minimo i tempi di generazione e i vincoli necessari per costruire le prove. Tra queste
funzioni, la funzione Poseidon è quella che ha ottenuto i risultati migliori e che verra utilizzata nella costruzione
del circuito per il protocollo RLN. Di seguito presento delle tabelle contenteti un confronto delle prestazioni tra le
più note funzione di Hash per le tecnologie zero knowledge e la funzione sh256, tabelle tratte da "POSEIDON: A New Hash
Function for Zero-Knowledge Proof Systems"\cite{cryptoeprint:2019-458}.Nelle tabelle sottostanti possiamo notare che ci
sono diverse configurazioni della funzione Poseidon. Infatti, una grande differenza tra le funzioni tradizionali e
Poseidon è la possibilità di scegliere il numero di iterazioni e la dimensione del campo finito su cui lavorare, in modo
da selezionare la versione più performante a seconda del caso.
\begin{figure}[H]
    \centering
    \includegraphics[width=15cm]{./chapters/2.rln-protocol/images/3.poseidon_comparison.png}
    \label{fig:poseidon_comparison}
    \captionsetup{justification=centering}
    \caption{Confronto con altri algoritmi Hash}
\end{figure}

\subsection{Nullifier}
I "nullifier" sono dei valori utilizzati per confermare o annullare operazioni. Nelle applicazioni che garantiscono l'anonimato, questi valori vengono spesso utilizzati per evitare il problema del "double-signaling", ovvero impedire che un utente utilizzi o esegua un'operazione che dovrebbe essere unica per ogni utente più di una volta. Questo problema può verificarsi perché, in assenza di un collegamento tra i dati e le identità degli utenti, non è possibile verificare se un utente ha già effettuato o completato una determinata operazione. Ad esempio, in un'applicazione elettorale è importante impedire che un singolo utente voti più di una volta, mentre in una blockchain è fondamentale evitare che la stessa moneta venga spesa più volte. Per evitare questo problema si utilizzano i nullifier ovvero valori univoci legati all'operazione che riamangono privati fino al momento dell'effetuazione dell'eoperazione e una volta attauta vengono resi pubblici, e salvati in modo da poterli consultare successivamente.
\begin{figure}[H]
    \centering
    \includegraphics[width=10cm]{./chapters/2.rln-protocol/images/4.nullifier.png}
    \label{fig:nullifier}
    \captionsetup{justification=centering}
    \caption{Rappresentazione della creazione di un nullifier, tratta da \cite{some-ways-to-use-zk-snarks-for-privacy}}
\end{figure}
Dalla descrizione fornita potrebbe subito venire in mente il concetto di "rate-limiting" e l'uso dei "nullifier" per attuare questa procedura in modo anonimo. In effetti, è possibile limitare gli utenti utilizzando questa metodologia, ma non si potrà rivelare l'identità dell'utente malintenzionato che, in questo modo, potrebbe riprovare indisturbato ad attacare il sitema.

Dopo aver visto le tecnologie utilizzate nel protocollo RLN, è possibile procedere con la descrizione delle sue fasi.
\begin{figure}[H]
    \centering
    \includegraphics[width=13cm]{./chapters/2.rln-protocol/images/5.rln_flow.png}
    \label{fig:rln_flow}
    \captionsetup{justification=centering}
    \caption{Diagramma funzionametno RLN, tratto da \cite{rln_doc}}
\end{figure}

\section{Registrazione}
La registrazione consiste nella prima fase del protocollo RLN, in questa fase gli utenti, generano un identity
commitment partendo dalla propria chiave privata, attraverso l'utilizzo della funzione hash Poseidon. Per semplificare
la spiegazione da qui in poi, utilizzeremo il simbolo $a_0$ per indicare la chiave privata.
$$identityCommitment = Poseidon(a_0)$$ in alcuni casi potremmo avere la necessità di elaborare maggiormente la creazione
dell'identity commitment per aumentare il livello di sicurezza o associare all'identity commitment anche una stake. Una
volta generato l'identity commitment l'utente viene inserito all'interno del albero di Merkle del sistema. A questo
punto l'utente sarà in grado di generare una "proof of membership" e quindi di interrogare il sistema. Per evitare
attacchi di tipo Sybil, è possibile implementare delle tecniche che vincolino l'inserimento dell'utente nell'albero al
rispetto di alcuni requisiti, come il possesso di identità digitali (CIE o SPID), un profilo social accreditato o ancora
un portafolgio di criptovalute.

Tra i progetti del gruppo PSE troviamo anche un progetto chiamato Interep\footnote{url{https://interep.link/}}, il cui
scopo è essenzialmente estrapolare la reputazione di un utente e inserire un identity commitment a lui associato
all'interno di alcuni gruppi di reputazione. Questi gruppi (alberi di Merkle) sono divisi in base al grado di
reputazione degli utenti. Attualmente il progetto Interep permette di estrapolare la reputazione di un utente dagli
account social di Github, Twitter e Reddit.
\section{Interazione}
Passata la fase di registrazione, gli utenti avranno la possibilità di effettuare richieste, provando di appartenere ai
membri del servizio, senza rivelare la loro identità utilizzando zk-SNARK. Ora ci chiediamo come possiamo implementare una regola di
rate-limiting del tipo: "Un utente non può fare più di $k$ richieste per un determinato lasso di tempo $e$ (epoca)"?

RLN utilizza l'algoritmo Shamir's Secret Sharing (SSS) che permette di suddividere un segreto in $n$ parti in cui
ciascuna parte del segreto non rivela nulla, ma se ne vengono combinate $k$ dove $k < n$ allora il segreto può essere
ricostruito. Ogni volta che l'utente fa una richiesta al sistema, rilascia una delle $n$ porzioni in cui la sua chiave
privata $a_0$ è stata divisa. In questo modo, se l'utente raggiungesse il valore di soglia $k$ imposto dalla regola di
rate-limiting il sistema sarebbe in grado di ricostruire $a_0$ svelando l'identità dell'utente in questione.

La procedura per dividere e ricostruire il segreto si basa ancora una volta sull'utilizzo dei polinomi, in particolare
sull'interpolazione di Lagrange. Il grado del polinomio da utilizzare per ricostruire la chiave privata a partire dai
suoi componenti, dipende strettamente dal numero di richieste che si desidera consentire. In particolare, per interpolare
(cioè ricostruire) un polinomio di grado $k$, abbiamo bisogno di almeno $k+1$ punti ($k+1$ richieste).

Vediamo un esempio di funzionamento dell'algoritmo, immaginiamo di voler applicare una regola di rate-limiting in cui :
"Un utente non può fare più di 1 richiesta al minuto". Per prima cosa costruiamo un nullifier, che ci servirà per
identificare i messaggi inviati all'interno di un epoca:
$$externalNullifier = Poseidon(epoch,rln\_identifier)$$ dove $epoch$ è l'epoca in cui è stato invito il messaggio e
$rln\_identifier$ è un valore univoco per tutta l'applicazione, questo valore viene utilizzato per proteggere gli utenti
che utilizzino la stessa chiave private in più servizi che apllicano RLN, infatti grazie a questo paramentro anche se si
usasse la stessa chiave privata per costruire il polinomio si otterrbbero valori differenti in fase di valutazione.
Proseguiamo con l'identificazione del polinomio che dovrà essere ricostruito dal verificatore (il sitema), il polinomio
in questione dovra essere di primo grado ($k=1$) in quanto voglaimos che con due richieste ($k+1$ punti) si possibile
ricostruirlo
$$ A(x) = a_1 * x + a_0$$ Notiamo che il polinomio valutato in 0 vale $a_0$ ovvero la nostra chiave privata, mentre
$a_1$ è definito come $a_1 = Poseidon(a_0, externalNullifier)$ che permette di variare il polinomio in base all'epoca in
cui viene fatta la richietsa, in questo caso specifico $a_1$ è il coefficente angolare della retta identificata dal
polinomio. Quando un utente invia una richiesta al sistema vengono calcolate due coordinate: $x = Poseidon(richiesta)$ e
$y=A(x)$ che identificano un punto sulla retta. Se un utente malitenzionato inviasse un altro nella setta epoca
otterrebbe le nuove cordinate $x_2 = Poseidon(richiseta_2)$ e $y_2=A(x_2)$ apprtenenti alla stessa retta e il sitema
sarebbe in grado di ricostruire il polinomio interpolando i due punti. Nel caso di un polinomio di primo grado, la
procedura di interpolazione è immediata
\begin{figure}[H]
    \centering
    \includegraphics[width=11cm]{./chapters/2.rln-protocol/images/6.a_0_interpolation.png}
    \label{fig:a_0_interpolation}
    \captionsetup{justification=centering}
    \caption{Grafico SSS pilinomio primo grado}
\end{figure}
\section{Punizione}
La fase di punizione o slashing varia molto a seconda del contesto applicativo e alla presenza o meno di una stake,
ma la logica è sepre 

\begin{figure}[H]
    \centering
    \includegraphics[width=12cm]{./chapters/2.rln-protocol/images/5.rln_flow.png}
    \label{fig:rln_flow}
    \captionsetup{justification=centering}
    \caption{Diagramma funzionametno RLN, tratto da \cite{rln_doc}}
\end{figure}



