\section{Strumenti}

\subsection{Alberi di Merkle}
Gli alberi di Merkle sono una particolare struttura ad albero che sfrutta le proprietà delle funzioni di Hash al fine di
ottimizzare le operazioni di ricerca e identificazione delle modifiche all'interno della collezione. La struttura degli
alberi di Merkle è caratterizzata dalle di foglie, che rappresentano i dati, a cui è stata applicata una funzione di
hashing, e dagli altri nodi dell'albero, che sono ottenuti applicando la funzione Hash ai loro figli, fino a raggiungere
la radice. Negli alberi di Merkle la radice è una funzione Hash che rappresenta univocamente la struttura. Per capire
meglio il concetto immaginiamo di avere un struttura ad albero binaria, e inseriamo all'interno della struttua gli
elementi 1,2,3,4,5 a cui applichimao una funzione Hash, a questo punto otterremo la seguente struttura:
\begin{figure}[H]
    \centering
    \includegraphics[width=15cm]{./chapters/2.rln-protocol/images/1.merkle_tree.png}
    \label{fig:merkle_tree}
    \captionsetup{justification=centering}
    \caption{Esempio struttura albero di Merkle binario}
\end{figure}
Possiamo notare che la radice dei Merkle tree possiede la proprietà di essere una sorta di impronta digitale della
struttura, in quanto qualsiasi modifica ai dati comporterebbe un cambiamento a cascata dei nodi fino alla radice stessa.
Questa proprietà costituisce un vantaggio significativo in termini di efficienza. Infatti, consente una verifica rapida
delle modifiche apportate alla struttura, rendendo gli alberi di Merkle estremamente utili nei campi decentralizzati e
nei file system condivisi, dove l'individuazione efficiente dei cambiamenti con poche informazioni è cruciale. Un'altra
proprietà altamente utile degli alberi di Merkle è la loro capacità di verificare l'appartenenza di un elemento alla
struttura in modo efficiente e senza la necesità di conoscere l'elemento in chairo. Nell'esempio precedente, se si
desidera verificare che l'elemento 4 appartenga alla struttura, è sufficiente utilizzare gli elementi H\_3, H\_12 e
H\_5000 e il valore H\_4, che non rivela alcuna informazion su l'elemento, per ricostruire tutti i nodi fino alla
radice. Una volta ottenuta la radice basterà confrontarla con la radice corretta per convicersi della presenza
dell'elemento o meno nella struttura.
\begin{figure}[H]
    \centering
    \includegraphics[width=15cm]{./chapters/2.rln-protocol/images/2.merkle_proof.png}
    \label{fig:merkle_proof}
    \captionsetup{justification=centering}
    \caption{Esempio Merkle proof}
\end{figure}
Questo processo viene chiamato "Merkle proof" o più in generale "proof of membership". Tale processo rappresenta
l'approccio che adotteremo per dimostrare la capacità di un utente già registrato e non rimosso di interagire con il
sistema nella fase di Interazione del protocollo RLN.

\subsection{Nuove funzioni di Hash}
Indubbiamente, una delle tecniche più utilizzate in crittografia sono le funzioni di Hash. Anche la tecnologia zk-SNARK
non può fare a meno di esse. Infatti, come abbiamo già visto in molte situazioni, l'utilizzo di metodi di hashing è
stato necessario per ridurre le dimensioni delle informazioni e per nasconderle.Tuttavia, quando si utilizzano le
tradizionali funzioni di Hash come la versione sha-256 nel campo delle zero knowledge proof, si verifica un problema.
Queste funzioni non sono state concepite per lavorare in domini di campi finiti e, pertanto, l'utilizzo di metodologie
come la ripetizione di operazioni bit-wise tende ad aumentare considerevolmente la dimensione dei vincoli del circuito
della prova che le implementa. Questo aumenta notevolmente il tempo e la dimensione richiesti per generare le prove.
Negli ultimi anni, per superare il problema descritto, sono state utilizzate nuove versioni di funzioni di Hash come
Pedersen, Mack e Poseidon. Queste funzioni sono state progettate specificamente per lavorare nei campi finiti e con
l'obiettivo di ridurre al minimo i tempi di generazione e i vincoli necessari per costruire le prove. Tra queste
funzioni, la funzione Poseidon è quella che ha ottenuto i risultati migliori e che verra utilizzata nella costruzione
del circuito per il protocollo RLN. Di seguito presento delle tabelle contenteti un confronto delle prestazioni tra le
più note funzione di Hash per le tecnologie zero knowledge e la funzione sh256, tabelle tratte da "POSEIDON: A New Hash
Function for Zero-Knowledge Proof Systems"\cite{cryptoeprint:2019-458}.Nelle tabelle sottostanti possiamo notare che ci
sono diverse configurazioni della funzione Poseidon. Infatti, una grande differenza tra le funzioni tradizionali e
Poseidon è la possibilità di scegliere il numero di iterazioni e la dimensione del campo finito su cui lavorare, in modo
da selezionare la versione più performante a seconda del caso.
\begin{figure}[H]
    \centering
    \includegraphics[width=15cm]{./chapters/2.rln-protocol/images/3.poseidon_comparison.png}
    \label{fig:poseidon_comparison}
    \captionsetup{justification=centering}
    \caption{Confronto con altri algoritmi Hash}
\end{figure}

\subsection{Nullifier}
I "nullifier" sono dei valori utilizzati per confermare o annullare operazioni. Nelle applicazioni che garantiscono l'anonimato, questi valori vengono spesso utilizzati per evitare il problema del "double-signaling", ovvero impedire che un utente utilizzi o esegua un'operazione che dovrebbe essere unica per ogni utente più di una volta. Questo problema può verificarsi perché, in assenza di un collegamento tra i dati e le identità degli utenti, non è possibile verificare se un utente ha già effettuato o completato una determinata operazione. Ad esempio, in un'applicazione elettorale è importante impedire che un singolo utente voti più di una volta, mentre in una blockchain è fondamentale evitare che la stessa moneta venga spesa più volte. Per evitare questo problema si utilizzano i nullifier ovvero valori univoci legati all'operazione che riamangono privati fino al momento dell'effetuazione dell'eoperazione e una volta attauta vengono resi pubblici, e salvati in modo da poterli consultare successivamente.
\begin{figure}[H]
    \centering
    \includegraphics[width=10cm]{./chapters/2.rln-protocol/images/4.nullifier.png}
    \label{fig:nullifier}
    \captionsetup{justification=centering}
    \caption{Rappresentazione della creazione di un nullifier, tratta da \cite{some-ways-to-use-zk-snarks-for-privacy}}
\end{figure}
Dalla descrizione fornita potrebbe subito venire in mente il concetto di "rate-limiting" e l'uso dei "nullifier" per attuare questa procedura in modo anonimo. In effetti, è possibile limitare gli utenti utilizzando questa metodologia, ma non si potrà rivelare l'identità dell'utente malintenzionato che, in questo modo, potrebbe riprovare indisturbato ad attacare il sitema.