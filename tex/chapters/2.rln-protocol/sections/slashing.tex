\section{Punizione}
L'ultima fase del protocollo consiste nella punizione dell'utente malintenzionato. Questa procedura è fortemente legata
al contesto applicativo. L'idea generale è che se sono presenti $k$ porzioni diverse per ricostruire
$a_0$ dalle coordinate $x$ e $y$, allora l'identità dell'utente può essere rivelata e l'utente può essere rimosso
dall'albero di Merkle ricostruendo il suo l'identity commitment. Questo comporta
l'azzeramento della foglia che contiene l'identity commitment dell'utente all'interno dell'albero. Inoltre, a seconda dell'applicazione,
la chiave privata può anche essere usata per sequestrare la stake fornita dall'utente.