\section{Registrazione}
La registrazione consiste nella prima fase del protocollo RLN, in questa fase gli utenti, previa la generazione di una chiave privata e ricavando da questa un "identity commitment" utilizzando la funzione hash Poseidon. Per semplificare la spiegazione da qui in poi, utilizzeremo il simbolo $a_0$ per indicare la chiave privata.
$$identityCommitment = Poseidon(a_0)$$
in alcuni casi potremmo avere la necessità di elaborare maggiormente la creazione dell'identity commitment per aumentare il livello di sicurezza o associare all'identity commitment anche una stake. Una volta generato l'identity commitment l'utente viene inserito all'interno del albero di Merkel del sistema. A questo punto l'utente sarà in grado di generare dell "proof of membership" e quindi di interrogare il sisteam. Per evitare attacchi di tipo Sybil, è posibile implementare dell'tecniche che vincolino l'inserimento dell'utente nell albero al rispetto di alcuni vincoli, come posseso identità digitali o di un profilo social accreditato o ancora un portafolgio di criptovalute.

Tra i progetti del gruppo PSE troviamo anche un progetto chiamato Interep, il cui scopo è essenzialemtene estrapolare la reputazione di un utente e inserire un identity commitment a lui associato all'inetrno di alcuni gruppi di reputazione. Questi gruppi (alberi di Merkel) sono divisi in base al grado di reputazione degli utenti. Attualemente il progetto Iterep permette di estrapolare la reputazione di un utente dagli acocunt socail di Github, Twitter e Reddit.