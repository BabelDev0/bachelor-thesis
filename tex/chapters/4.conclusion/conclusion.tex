\chapter*{Conclusioni}
\chaptermark{Conclusioni}
\addcontentsline{toc}{chapter}{Conclusioni}

Durante l'analisi della tesi, sono stati descritti numerosi concetti e tecnologie. Abbiamo iniziato parlando degli
attacchi DoS, e come questi rappresentino un tipologia di attacco informatico molto comune e altamente efficace. Per
difendersi da tali attacchi, sono state sviluppate diverse strategie. La maggior parte di queste si basa sull'analisi
del traffico di rete o sulle azioni degli utenti nel tempo, al fine di prevenire gli attacchi. Tra le numerose strategie
anti-DoS, ci siamo focalizzati sul rate-limiting, una tecnica che si basa sulla capacità di tenere traccia delle
richieste effettuate dagli utenti, al fine di limitare il numero di richieste consentite in un determinato intervallo di
tempo. Indipendentemente dalla strategia scelta per applicare una soluzione di rate-limiting, il principio alla base
consiste nel riuscire a identificare l'origine di ogni richiesta per poter associare ogni messaggio al suo mittente e
poterlo cosi limitare. Ciò fa si che questa, come tante altre strategie anti Dos, non lavorino bene in ambiente anonimo,
ovvero in ambienti dove poter identificare l'origine di ogni richiesta è difficile o impossibile, la spiegazione è
immediata se non so chi si comporta male non posso punirlo.

Per risolvere questo problema, durante la lettura della tesi abbiamo introdotto, una delle tecnologie più interessanti
degli ultimi anni in ambito di crittografia e teoria dell'informazione, ovvero le Zero Knowledge Proof (ZKP) e nello
specifico la loro versione più efficiente e non interattiva zk-SNARK. Questa tecnologia grazie a svariati principi
matematici e teorici consente la creazione di prove "Succint", ovvero di dimensioni molto inferiori alla complessità
dell'affermazione da dimostrare e "Non-interactive", ovvero senza la necessità che dimostratore o verificatore si
scambino messaggi per validarle. Queste caratteristiche rendono zk-SNARK uno strumento molto utile e versatile. L'idea
del protocollo RLN nasce proprio dalla possibilità di applicare, tramite alcuni costrutti ausiliari, la tecnologia
zk-SNARK per creare un protocollo di rate-imiting efficace e performante in ambiente anonimo. Il prototipo sviluppato e
esposto nel capitolo precedente dimostra come questo protocollo, compia abbondatemente il suo dovere, permettendo al
Server di limitare i Client senza avere alcuna nozione sulla loro identità e lo fa in modo non interattivo ed efficiente
con dimensioni e tempi ridotti.

Tuttavia, dalla descrizione del prototipo e dai capitoli precedenti, è possibile notare come il protocollo RLN sia
ancora non maturo, in quanto eredita molti degli svantaggi della tecnologia zk-SNARK. Infatti, come evidenziato nel
capitolo dedicato alla tecnologia, il processo di sviluppo dei circuiti e la generazione delle prove è molto complesso e
richiede un notevole dispendio di risorse. Inoltre, la robustezza della tecnologia, dipende strettamente da cerimonie
di trusted set-up complesse e che richiedono la partecipazione di molti utenti, il che ne limita sia l'efficacia che la
distribuzione. Recenti evoluzioni come PLONK, STARK e Bulletproof, hanno l'obbiettivo di migliorare alcuni aspetti come la scalabilità,
la sicurezza e la complessità della tecnologia zk-SNARK, aprendo nuove opportunità di applicazione. Ciò nonostante la ricerca per una
soluzione ottimale è ancora in corso in quanto queste tecnologie, pur non basandosi su trusted set-up, non riescono a
competere in termini di velocità e dimensioni delle prove con la tecnologia zk-SNARK.

In conclusione, la tecnologia zk-SNARK e ZKP in generale è una delle più promettenti innovazioni nel campo della
crittografia e della privacy. Grazie alla sua capacità di dimostrare la veridicità di informazioni senza rivelarne il
contenuto, questa tecnologia ha il potenziale per rivoluzionare molte industrie, come ad esempio quella finanziaria o
quella sanitaria. Ci sono ancora alcune sfide da affrontare, come la riduzione dei costi computazionali e la miglior
comprensione delle implicazioni della tecnologia sulla privacy e sulla sicurezza.

In ogni caso, il protocollo RLN riesce con efficacia ad applicare un sistema di rate-limiting a livello applicazione, in
un ambiente anonimo ed è indubbio che zk-SNARK rappresenti una tecnologia di grande interesse e potenziale, che nel
prossimo futuro, ci si aspetta, avrà un impatto significativo su molte aree della vita quotidiana.