\chapter{Prototipo}
\chaptermark{Prototipo}
\addcontentsline{toc}{chapter}{Prototipo}

In questo capitolo, presenteremo le parti significative di un prototipo di applicazione che utilizza il protocollo RLN e
la tecnologia zk-SNARK  per applicare una regola di rate-limiting in un sistema centralizzato, dove i Client
interagiscono con il Server in completo anonimato.

Al fine di concentrarci sulle dinamiche del protocollo per comprenderne il funzionamento a pieno, ho scelto di
semplificare la parte strutturale del prototipo; questa è composta da un Server che espone un servizio di registrazione
e di interazione, e alcuni Client che comunicano con il Server attraverso un canale di comunicazione basato su socket.

Si tenga presente che il protocollo non ha pretese di essere un progetto realmente utilizzabile, ma piuttosto di essere
un utile strumento per coloro che desiderano acquisire familiarità con le tecnologie zero knowledge e applicarle
concretamente. Ciò nondimeno sarà possibile valutare i benefici e le criticità che queste tecnologie portano in campo,
mediante il calcolo delle prestazioni e l'analisi del codice.

è possibile consultare il codice sorgente del prototipo su
GitHub\footnote{\url{https://github.com/BabelDev0/bachelor-thesis/tree/master/prototype}}

\section{Tecnologie e struttura}
Per lo sviluppo del prototipo, ho attuato la scelta di utilizzare il linguaggio di programmazione
TypeScript\footnote{\url{thttps://www.typescriptlang.org/}}. Questa decisione è stata presa in considerazione delle
specifiche necessità del progetto, e in particolare delle librerie sviluppate e attive per la tecnologia zk-SNARK e per
il protocollo RLN. La struttura del progetto è una monorepo (unica repository con molteplici progetti) che contiene sia
i dati e il codice del Server e del Client. Entrambi i componenti, utilizzano
Node.js\footnote{\url{https://nodejs.org/en}} come ambiente di esecuzione. Infatti dopo la compilazione del codice
TypeScript in JavaScript, il codice viene eseguito tramite il runtime Node.js. Le librerie utilizzate sono le seguenti:
\begin{itemize}
    \item \textbf{Circom 2\footnote{\url{https://docs.circom.io/}}}: usata per la generazione e la compilazione dei
    circuiti del progetto. Circom consiste in un compilatore Rust per i circuiti scritti in linguaggio Circom (omonimo
    della libreria). Inoltre, comprende anche diversi template di circuiti che assolvono specifiche funzioni molto utili,
    come l'implementazione della funzione di hash Poseidon.
    \item \textbf{Socket.IO\footnote{\url{https://socket.io/}}}: è una libreria JavaScript che consente di creare e
    gestire connessioni in tempo reale tra il Client e Server. Si basa sul protocollo WebSocket e permette di creare una
    comunicazione bidirezionale tra Client e Server implementando anche un sistema basato sugli eventi, ispirato alla
    classe EventEmitter\footnote{\url{https://nodejs.dev/en/learn/the-nodejs-event-emitter/}} di Node.js
    \item \textbf{RLN Circuits\footnote{\url{https://github.com/privacy-scaling-explorations/rln}}}: in questa libreria
    sono preseti i circuiti che implementano le funzioni principali del protocollo RLN. In particolare, sono presenti i
    circuiti per la registrazione, la verifica, la punizione e alcuni circuiti di supporto.
    \item \textbf{RLNjs\footnote{\url{https://github.com/Rate-Limiting-Nullifier/rlnjs}}}: è una libreria scritta in
    TypeScript che implementa la logica del protocollo RLN. Sfrutta i circuiti messi a disposizione dalla libreria
    precedente per implementare la logica di verifica e generazione delle prove.
\end{itemize}

Inoltre, ho utilizzato il package manager ufficiale di Node.js, npm\footnote{\url{https://www.npmjs.com/}}, per
installare rapidamente e facilmente le librerie e i moduli di cui ho avuto bisogno e
Eslint\footnote{\url{https://eslint.org/uno}}  strumento di analisi del codice statico che aiuta a individuare e
correggere errori e eventuali vulnerabilità del codice.

\section{Circuiti}
Come spiegato nella sezione relativa alla proprietà Succinct di zk-SNARK per poter creare un programma che usi questa
tecnologia, abbiamo bisogno di formulare le sezioni di logica del programma, che vogliamo dimostrare attraverso dei
circuiti algebrici. Le parti significative del codice di cui vogliamo creare una Zero Knowledge proof sono
l'appartenenza  all'albero di Merkle e la corretta costruzione delle porzione di segreto che ogni Client rilascia al
Server durante l'interazione. Per questo motivo abbiamo bisogno di due circuiti:\clearpage
\begin{enumerate}
    \item \textbf{Proof of membership}:
    \begin{figure}[H]
        \centering
        \includegraphics[width=12cm]{./chapters/3.poc/images/1.merkle_proof.png}
        \label{fig:merkle_proof_code}
        \captionsetup{justification=centering}
        \caption{Codice circuito MerkleTreeInclusionProof}
    \end{figure}
    Nel codice in figura possiamo vedere il circuito scritto in codice Circom che implementa la ricerca di un elemento
    all'interno di un albero di Merkle, il circuito prende in ingresso tre parametri: la profondità dell' albero, il
    valore della foglia da cercare, il percorso binario (dove 0 indica il ramo di sinistra e 1 indica il ramo di destra)
    che bisogna seguire partendo dalla radice per raggiungere la foglia e le altre foglie dell'albero. Il circuito
    restituisce come risultato la radice dell'albero. I componenti $hashLeftRight()$ e $MultiMux1()$ sono due
    componenti ausiliari che servono rispettivamente ad applicare la funzione hash ai nodi figli e a selezionare i nodi
    corretti in base al livello dell'albero in cui ci troviamo.

    Nella prima sezione di codice vediamo due funzioni fondamentali del protocollo RLN, ovvero la funzione
    $CalculateIdentityCommitment()$ che permette di verificare che l'identity commitment venga effettivamente calcolato
    a partire dalla chiave privata e $CalculateExternalNullifier()$ che è quel circuito che ci permette di ottenere un
    nulliffier univoco per singola epoca.\clearpage
    \item \textbf{Verifica delle porzioni del segreto}:
    \begin{figure}[H]
        \centering
        \includegraphics[width=11cm]{./chapters/3.poc/images/2.1.verify_shares.png}
        \label{fig:1.verify_shares}
        \captionsetup{justification=centering}
        \caption{Codice circuito CalculateIdentityCommitment e CalculateExternalNullifier}
    \end{figure}
    
    \begin{figure}[H]
        \centering
        \includegraphics[width=11cm]{./chapters/3.poc/images/2.2.verify_shares.png}
        \label{fig:2.verify_shares}
        \captionsetup{justification=centering}
        \caption{Codice circuito CalculateSecretShare}
    \end{figure}
    Mentre in questa sezione vediamo ancora una funzione chiamata CalculateA1() che permette di ottenere il valore $a_1$
    responsabile della variazione del polinomio a seconda dell' external\_nullifier e della chiave privata, seguendo la
    formula vista in precedenza $a_1 = Poseidon(a_0, externalNullifier)$ e in fine vediamo una porzione di codice dove
    tutte le funzioni precedenti vengono utilizzate per verificare che il segreto sia stato costruito correttamente.
\end{enumerate}

Ricordo che il codice finora non rappresenta la logica del protocollo, ma solamente i circuiti algebrici che verranno
trasformati in vincoli, per controllare che il Client rispetti pur rimanendo anonimo le regole del servizio.

Per ottenere i file che verranno utilizzati da Server e Client per provare e verificare le dimostrazioni, bisogna
compilare i circuiti utilizzando Circom. Alla fine della fase di compilazione che si divide in trasformazione da
circuito a QAP e successiva fase di trusted set-up (che ovviamente in questo caso è stata svolta con un unico
partecipante), otteniamo i seguenti file:
\begin{itemize}
    \item \textbf{rln\_final.zkey}: file che contiene i parametri crittografici e i paramentri privati che consentono a
    un verificatore di controllare la validità di una prova senza rivelare alcuna informazione sui dati sottostanti.
    \item \textbf{rln.wasm}: è una versione compilata del circuito che può essere eseguita in un browser web utilizzando
    WebAssembly. Questo file è generato dal codice Circom e contiene la logica del circuito in un formato binario che
    può essere eseguito.
    \item \textbf{verification\_key.json}: contiene una chiave pubblica relativa al circuito, che può essere usata da un
    verificatore per controllare la validità di una prova. \end{itemize}

\section{Server}
La struttura del Server è semplice: si tratta di un progetto Node.js che utilizza la libreria Socket.IO per gestire la
comunicazione con i Client. Il Server è composto da due classi principali: "server.ts" e "type.ts". La prima classe
contiene la logica applicativa e implementa le primitive della libreria RLNjs per verificare le prove dei Client. La
seconda classe è un file che definisce degli enum, utilizzati per identificare lo stato e gli eventi di comunicazione
tra il Server e i Client. Questo file, insieme ai file di configurazione del circuito, è presente sia nel progetto
Server che in quello Client.

Il Server attende la comunicazione con un Client sulla porta 3000 e rimane in attesa fino a quando non riceve un evento.
Gli eventi possibili sono definiti nel file "type.ts" e quelli accettati dal Server sono EventType.REGISTER e
EventType.INTERACTION. Questi eventi sono utilizzati rispettivamente per gestire la registrazione e l'interazione con
gli utenti. Durante la fase di registrazione, il Server inserisce gli utenti nell'albero di Merkle, chiamato "registry"
nel programma, e notifica lo stato della registrazione, che può essere uno dei seguenti: ALREADY\_REGISTERED, BANNED o
VALID. 
\begin{figure}[H]
    \centering
    \includegraphics[width=9cm]{./chapters/3.poc/images/3.1.Server.png}
    \label{fig:1.Server}
    \captionsetup{justification=centering}
    \caption{Codice Server REGISTER}
\end{figure}
Mentre per quanto riguarda la fase di interazione, il Server si occupa prima di tutto di controllare che l'utente abbia
inviato una prova valida. Questo controllo avviene verificando che la radice del suo registry collimi con quella del
Server e che la generazione della prova abbia seguito i vincoli specificati dal circuito. Successivamente, il Server
controlla se il messaggio inviato rispetta le regole anti-spam. Se ciò non avviene, il Server rimuove l'utente e
sincronizza i registry dei suoi Client inviando un messaggio di broadcast a tutti i membri. Nelle fasi successive,
approfondiremo i tempi necessari per lo svolgimento di queste funzioni.
\begin{figure}[H]
    \centering
    \includegraphics[width=10cm]{./chapters/3.poc/images/3.2.Server.png}
    \label{fig:2.Server}
    \captionsetup{justification=centering}
    \caption{Codice Server INTERACTION}
\end{figure}

\section{Client}
Il Client presenta una struttura molto simile a quella del Server, utilizza la libreria Socket.IO-client per gestire la
comunicazione. Inoltre, dispone di un file "type.ts" analogo a quello del server per gestire gli eventi di comunicazione
e conserva all'interno del progetto i file relativi alla generazione del circuito. Tuttavia, a differenza del server, il
Client utilizza tali file per generare le prove e non per verificarle. Gli eventi rilevanti per il Client sono
EventType.REGISTER, EventType.INTERACTION, EventType.USER\_REGISTERED e EventType.USER\_SLASHED. Gli ultimi due sono
generati dal Server e servono a sincronizzare gli alberi di merkle degli utenti a seguito della registrazione o
rimozione di un nuovo membro, infatti per poter generare prove valide, gli utenti devono possedere la stessa versione
della struttura posseduta dal Server altrimenti il primo controllo sulla radice dei registri, non andrebbe a buon fine.
\begin{figure}[H]
    \centering
    \includegraphics[width=11cm]{./chapters/3.poc/images/4.1.client.png}
    \label{fig:1.client}
    \captionsetup{justification=centering}
    \caption{Codice Client refreshRegistry}
\end{figure}
Per ottimizzare la fase di interazione la generazione della prova di appartenenza al registro viene effettuata ad ogni
sincronizzazione. La fase di registrazione è abbastanza banale in quanto è costituita solamente dall'invio dell'identity
commitment al server e dall'attesa del relativo aknowledgement. Mentre la fase di interazione è più articolata, in
quanto è durante questa fase che il Client genera la dimostrazione di appartenere all'albero e di avere generato le
porzioni di chiave da rilasciare, in modo corretto. In questa fase, per testare le funzionalità di rate-limiting, ho
incluso la possibilità per l'utente di selezionare l'opzione "dos". Grazie a questa opzione, l'utente potrà provare ad
inviare 100 messaggi nello stesso intervallo di tempo. Prima di interagire con il sistema l'utente genera una prova
utilizzando: il segnale che una volta applicata la funzione hash rappresenterà la coordinata $x$, la merkleProof e
l'externalNullifier. Le altre informazioni necessarie come la chiave privata o l'rln\_identifier vengono ricavate da un
istanza della libreria RLNjs creata durante le fasi iniziali del Client.
\begin{figure}[H]
    \centering
    \includegraphics[width=11cm]{./chapters/3.poc/images/4.2.client.png}
    \label{fig:2.client}
    \captionsetup{justification=centering}
    \caption{Codice Client INTERACTION}
\end{figure}

\section{Prestazioni}
Di seguito riporto alcuni dati significativi ottenuti dall'esecuzione del prototipo. I test sono stati effettuati su un
computer con le seguenti caratteristiche:
\begin{itemize}
    \item CPU: Apple silicon M1
    \item RAM: 8GB
    \item OS: MacOS Ventura 13.0
\end{itemize}

Inizieremo analizzando il numero di vincoli che costituiscono i circuiti, per la generazione delle prove e le loro
dimensioni. È importante ricordare che i circuiti sono stati generati utilizzando la libreria Circom 2. Per i test che
verranno presentati, sono state utilizzate tre diverse profondità degli alberi di Merkle: 16, 24 e 32 livelli, che
corrispondono rispettivamente ad un massimo di $2^{16}$, $2^{24}$ e $2^{32}$ utenti.
\begin{figure}[H]
    \centering
    \includegraphics[width=15cm]{./chapters/3.poc/images/5.1.bench.png}
    \label{fig:1.bench}
    \captionsetup{justification=centering}
    \caption{Prestazioni circuiti sui vincoli}
\end{figure}
I dati più significativi sono il numero di vincoli, le dimensioni dei file rln\_final.zkey e rln.wasm. Infatti, se a un
primo impatto i numeri dei vincoli generati dai circuiti possono sembrare molto elevati, in realtà le dimensioni dei
file sono notevolmente ridotte. Ragionando sul fatto che un albero di Merkle di profondità 30, che utilizza una
funzione di hash con output 128 bit, può arrivare pesare decine di GB, una prova che pesi solamente qualche mega si può
definire Succinct. 

Di seguito invece mostriamo i tempi di esecuzione:
\begin{figure}[H]
    \centering
    \includegraphics[width=15cm]{./chapters/3.poc/images/5.2.bench.png}
    \label{fig:2.bench}
    \captionsetup{justification=centering}
    \caption{Prestazioni circuiti sui tempi}
\end{figure}

Qui i dati più significativi sono i tempi di generazione e verifica delle prove, in particolare la verifica che rimane
praticamente costante. La colonna con indice "Verifica radice dell'albero" rappresenta la velocità con cui il sistema
riesce a verificare se la radice dell'albero con cui è stata generata la prova è uguale a quella in possesso al Server.